\documentclass{article}
\usepackage[english]{babel}
\usepackage[letterpaper,top=2cm,bottom=2cm,left=3cm,right=3cm,marginparwidth=1.75cm]{geometry}
% Useful packages
\usepackage{amsmath}
\usepackage{graphicx}
\usepackage{float}
\usepackage[colorlinks=true, allcolors=blue]{hyperref}



\title{Convection Diffusion with Linear Production}
\author{ }


\begin{document}
	
\maketitle	
The Convection - Diffusion - Production equation is one of the most frequently implemented equations for the description of physical phenomena across almost every scientific domain. For a homogeneous material with a linear source \ sink term, the general form of the equation is: phgh linear wiki

\begin{equation}
	\alpha \frac{\partial \phi}{\partial t} = 
	\beta \boldsymbol{\nabla}^2 \boldsymbol{\phi}
	- \gamma \nabla \cdot (\boldsymbol{v} \boldsymbol{\phi})
	+ R(\boldsymbol{\phi})
\end{equation}	

\begin{itemize}
	\item $R(\boldsymbol{\phi}) = \delta \boldsymbol{\phi} + \epsilon$ is a linear source $(R(\boldsymbol{\phi}) > 0)$ or a linear sink ($R(\boldsymbol{\phi}) < 0)$	
	\item $\alpha$ is the $\boldsymbol{capacity \  coefficient}$. It scales the time dependent term.
	\item $\beta$  is the $\boldsymbol{diffusion \  coefficient}$. Generally diffusion is associated with transport phenomena that occur due to the medium molecular properties. It measures how easily does the property of interest disperse  in the host medium. 
	\item $\gamma$ is the $\boldsymbol{convection\  coefficient}$. Convection is associated with transport phenomena due to a flow of velocity $\boldsymbol{v}$ inside and/or at the boundaries of the control volume.
	\item $\delta$ is the $\boldsymbol{dependent \ source \ term \ coefficient}$.
	\item $\epsilon$ is the $\boldsymbol{independent \ source \ term \ coefficient}$.
\end{itemize}


\section{Weak Form - Spatial Discretization}


The physical space is discretized in N elements with M nodes each. We assume that the solution takes the form:
\begin{equation} \label{eq:TiNi}
	\boldsymbol{\Phi} \cong{\boldsymbol {N\phi} = [ N_1,...,N_M] \begin{bmatrix} \phi_1\\ \vdots \\ \phi_M \end{bmatrix} = N_i\phi_i}
\end{equation}

where $N_i$ are the weight functions and $\phi_i$ are the nodal parameters. By substituting \eqref{eq:TiNi} in \eqref{eq:eq} we deduce:

\begin {equation}\label{eq:discrete1}
\alpha \frac{\partial}{\partial t}{\boldsymbol {N \phi}} =
 \beta \boldsymbol {\nabla^2 N\phi}
- \gamma \boldsymbol{v \nabla N \phi}
+ \delta \boldsymbol {N \phi}
+ \epsilon
\end {equation}


Multiply with weight functions vector and integrate over the element volume:
\begin {equation}\label{eq:discrete2}
\alpha \iiint {\boldsymbol{N^T} \frac{\partial}{\partial t}{\boldsymbol {N\phi}}dV} =
\beta \iiint{\boldsymbol{N^T \nabla^2 N \phi} dV
-\gamma \boldsymbol{u} \iiint{\boldsymbol{N^T} \nabla \boldsymbol{N\phi}} dV
+\delta \iiint{\boldsymbol{N^T} \boldsymbol{N\phi}} dV
+ \epsilon \iiint{\boldsymbol{N^T}dV}
	\end {equation}
	
	
Nodal parameter $\phi_i$ is independent of space and time so by integrating by parts and ignoring the boundary integrals \eqref{eq:discrete2} can be written as:


\begin {equation}\label{eq:discrete3}
	\alpha \frac{\partial \boldsymbol{\phi}}{\partial t}\iiint {{\boldsymbol {N^TN}}dV} =
	\beta \boldsymbol{\phi}\iiint{\boldsymbol {\nabla (N^T) \nabla N}dV}
	-\gamma \boldsymbol{v \phi} \iiint{\boldsymbol{N^T \nabla N}} dV
	+\delta \phi \iiint{\boldsymbol{N^T} \boldsymbol{N\phi}} dV
	+ \epsilon \iiint{\boldsymbol{N^T}dV}
\end {equation}

\eqref{eq:discrete3} can be simplified to:
\begin{equation}
	\alpha\boldsymbol{M} \frac{\partial \boldsymbol \phi}{\partial t}
	+\beta \boldsymbol{C \phi}
	- \gamma \boldsymbol{K\phi} = 
	\boldsymbol{F}
\end{equation}
where 
\begin{equation} \nonumber
	\boldsymbol{M} = \iint {{\boldsymbol {N^TN}}dA} 
\end{equation}

\begin{equation} \nonumber
	\boldsymbol{C} = \iint {{\boldsymbol {N^T \nabla N}dA} 
	\end{equation}
	
	\begin{equation} \nonumber
		\boldsymbol{K} = \iint{\boldsymbol {\nabla N^T\nabla N}dA}
	\end{equation}
	
	\begin{equation} \nonumber
		\boldsymbol{F} = f\iint {{\boldsymbol {N^T}}dA} 
	\end{equation}






\end{document}	
